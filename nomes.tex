%%------------------------------------------------------------------------------
%%----------------Variaveis para o LaTeX no Relatório---------------------------
\newcommand{\universidade}{Universidade de Bras\'ilia - Campus Gama}
\newcommand{\hell}{Nome do Arquivo}		%nome do arquivo
\newcommand{\autor}{\ziul}							%criador do pdf - só observado nos detalhes do pdf
\newcommand{\assunto}{Nome da Materia/Experimento}			%Nome do experimento
\newcommand{\ver}{Titulo~- \assunto}						%numeração do experimento
\newcommand{\professor}{Professor}

%%------------------------------------------------------------------------------
%%------------------------Detalhes para serem revistos--------------------------
\newcommand{\keyw}{Palavras,  Chaves, \assunto}	%Keywords
\newcommand{\entrega}{\today}		%data de entrega do relatório


%%------------------------------------------------------------------------------
%%------------------------------Nomes-------------------------------------------
\newcommand{\luiz}{\href{http://lattes.cnpq.br/1109478949026592}{Luiz Fernando Gomes de Oliveira}}	%Para link
\newcommand{\ziul}{{Luiz Fernando Gomes de Oliveira}}					%Sem link
\newcommand{\luizmatricula}{10/46969}
\newcommand{\eluiz}{ziuloliveira@gmail.com}

%----------------------Foto e bibliografia do(s) autor(es)----------------------
%%------------------------------------------------------------------------------
%					  Deve de ser COLADO ao final do texto
%%------------------------------------------------------------------------------
%	\begin{IEEEbiography}[{\includegraphics[width=1in,height=1.25in,clip,keepaspectratio]{./fts/luiz}}]{\luiz}\label{luiz}
%	É, sou eu. Aparecendo aqui só de brinks. Meio que trollando um relatório.
%	\end{IEEEbiography}
%
%	\begin{IEEEbiography}[{\includegraphics[width=1in,height=1.25in,clip,keepaspectratio]{ffuu}}]{Helbert Junior}\label{panda}
%	Pow, eu tinha que aparecer também né?
%	\end{IEEEbiography}
%%------------------------------------------------------------------------------

%%------------------------------------------------------------------------------
%%-----------------------------Organização--------------------------------------
\makeatletter
\@ifclassloaded{scrartcl}% Slide
{
	\newcommand{\names}{\luiz}
	\newcommand{\namecapa}{\ziul \\ \ziul}
	%\newcommand{\allmatriculas}{\luizmatricula,\canelamatricula}
}
{
	\newcommand{\names}{\luiz~- \luizmatricula }
}
\makeatother


\newcommand{\cor}{vermelho}
\newcommand{\comprimeto}{20m}
\title{\ver}

\makeatletter
\@ifclassloaded{coursepaper}
{
	\author{\ziul}
	\studentnumber{\luizmatricula} %Matricula
	\instructor{\professor}
	\coursenumber{\hell}
	\college{\universidade}
	\date{\entrega}

	%% Redefinindo Titulo
	\def\@maketitle{%
	  %\newpage
	  %\null
	  \begin{center}
	  \includegraphics[width=\textwidth]{./fga.jpg}%	
	  \end{center}
	  
	  \vskip 2em%
	  \begin{flushleft}%
	  	\begin{spacing}{1}%
	    {\sffamily \LARGE \@title \par}%
	    \vskip 1em%
	    {\sffamily \large\lineskip .75em\@author ~-~ \@studentnumber}\\%
	    %\vskip 1em%
	    \sffamily \@date\\
	    \vskip 1em%
	    \sffamily \@college\\%
	    \sffamily \@coursenumber\hskip 6pt\\%
	    \sffamily Professor: \@instructor%
	    \par%
	    \end{spacing}%
	  \end{flushleft}\hrule\vskip 1em\par
	  \par
	  \vskip 1.5em}

 	\usepackage{fancyhdr}
 	\fancyhead{\ver}
 	\fancyfoot{ }
 	\fancyhead[LE,RO]{Página \thepage ~de \pageref{LastPage}}
 	\fancyhead[LO]{\hell}
 	\fancyhead[RE]{\slshape\leftmark}
 	\pagestyle{fancy}

}
% end coursepaper

\@ifclassloaded{exam}
{%
	\printanswers %apresentar respostas
	\renewcommand{\solutiontitle}{\noindent\textbf{Resposta:}\par\noindent}
	\usepackage{setspace}

	\pagestyle{head}
	\firstpageheader{}{\includegraphics[width=\textwidth]{./fga.jpg}}{}
	\runningheader{\hell}{Página \thepage\ de \numpages}{\entrega}
	\runningheadrule

	%% Redefinindo Titulo
	\def\@maketitle{%
	  \newpage
	  \null
	  \includegraphics[width=\textwidth]{./fga.jpg}%
	  \vskip 2em%
	  \begin{flushleft}%
	  	\begin{spacing}{1}%
	    {\sffamily \LARGE \@title \par}%
	    \vskip 1em%
	    {\sffamily \large\lineskip .75em\ziul ~-~ \luizmatricula}\\%
	    %\vskip 1em%
	    \sffamily \@date\\
	    \vskip 1em%
	    \sffamily \universidade\\%
	    \sffamily \hell\hskip 6pt%
				\\%
	    \sffamily Professor: \professor%
	    \par%
	    \end{spacing}%
	  \end{flushleft}\hrule\vskip 1em\par
	  \par
	  \vskip 1.5em}
}
%end exam
{
	% \usepackage{setspace}
	\title{%\includegraphics[width=18cm]{./fts/cap.png}\\
	\hell\\\ver}
	\@ifclassloaded{IEEEtran}
	{
		%%------------------------------------------------------------------------------
		%%	Resolveu o problema de a linguagem não estar declarada
		\makeatletter
		\def\markboth#1#2{\def\leftmark{\@IEEEcompsoconly{\sffamily}\MakeUppercase{\protect#1}}%
		\def\rightmark{\@IEEEcompsoconly{\sffamily}\MakeUppercase{\protect#2}}}
		\makeatother
		\newcommand{\ieee}{true}
		\newcommand{\review}{
			\IEEEcompsoctitleabstractindextext{%
			\begin{abstract}
				\input{abstract}
			\end{abstract}
			%%-------------------------------------
			\begin{IEEEkeywords} %palavras chaves
				\centering\keyw %,\LaTeX.
			\end{IEEEkeywords}}
			
			\thispagestyle{empty}
			\IEEEdisplaynotcompsoctitleabstractindextext
			\IEEEpeerreviewmaketitle
		}
		\author{
			\ifthenelse{\equal{\journal}{true}}{
				\names\\
			}{
			%\thanks{Revisado em \today.}
			\IEEEauthorblockN{\luiz}
			\IEEEauthorblockA{Matricula: \luizmatricula\\E-mail: \eluiz}
%			\and
%			\IEEEauthorblockN{\canela}
%			\IEEEauthorblockA{Matricula: \canelamatricula\\E-mail: \ecanela}
			}
		}
	}
	{
		\newcommand{\ieee}{false}
		\author{\names}
		\newcommand{\review}{
		\begin{abstract}
				\input{abstract}
		\end{abstract}
		}
		
	}
}
\makeatother

%%------------------------------------------------------------------------------
% Cabeçalho das páginas, se tiver no modelo
\markboth{Universidade de Bras\'ilia - Campus Gama - FGA, \entrega}
{Shell \MakeLowercase{\textit{et al.}}: \ver}
